\documentclass[sigconf]{acmart}

\usepackage{graphicx}
\usepackage{hyperref}
\usepackage{todonotes}

\usepackage{endfloat}
\renewcommand{\efloatseparator}{\mbox{}} % no new page between figures

\usepackage{booktabs} % For formal tables

\settopmatter{printacmref=false} % Removes citation information below abstract
\renewcommand\footnotetextcopyrightpermission[1]{} % removes footnote with conference information in first column
\pagestyle{plain} % removes running headers

\newcommand{\TODO}[1]{\todo[inline]{#1}}


\begin{document}
\title{Big Data Application in Precision Medicine and Pharmacogenomicsn}


\author{Budhaditya Roy}
\orcid{hid 348}
\affiliation{%
  \institution{Indiana University}
  \streetaddress{School of Information and Computing}
  \city{Bloomington} 
  \state{IN} 
  \postcode{47040}
}
\email{royb@indiana.edu}

% The default list of authors is too long for headers}
\renewcommand{\shortauthors}{B.Roy.}


\begin{abstract}
This article focuses on the impending impact of big data analytics improving health, preventing and detecting illness at a preliminary stage of illness and personalize interferences. The complexity and diversity of biological data are pouring the need of big data analytics and how it is applied in biological field especially in Pharmacogenetics, personalized/precision medicine. Big data is particularly very useful in the healthcare industry as a whole for its data handling intensive nature. Over the past decade, electronic health records (EHR) have become an extensively accepted in hospitals and clinics worldwide and the amount of information is generally daily from a single patient is increasing day by day. Important clinical acquaintance and a deeper understanding of patient disease patterns can be deliberate from such data. It will help to improve patient care as well improve efficiency of patient care and disease prevention. There are few applications pointed out to be effective using big data such as Healthcare data solutions and big data in cancer therapy, continuous monitoring of patients symptoms, healthcare intelligence, fraud prevention and detection. Many people heard about the proposition of precision medicine in State of Union speech of President Obama in 2012. Since then the revolutionary process of precision medicine started to grow rapidly in healthcare industry. On January 30, on the same Precision based medical initiatives, the Obama administration exposed facts about the Precision Medicine tentative plans. Threw with a 215 million dolar investment in the US President’s 2016 budget,, the Precision Medicine Initiative will product a new model of patient driven research that eventually support delivering the right treatment to the right patient at the right time\cite{editor01}. On March 11, 2015, it is reported that China is planning to invest 60 billion Yuan almost 10 billion) in precision medicine (20 billion from the Central Government and the remaining 40 billion from local governments and companies) before 2030 [\cite{editor01}]. There is a similar necessity of big data application to this latest emergence of biomedical domain. . Big data in precision medicine is the most widely used methods in precision and personalized medicine which is a life changing event in healthcare industry. Personalized medicine or called as Precision medicine is product and services that leverage the science of genomics and proteomics and take advantage of on the trends concerning wellness to enable preventive care. By using big data analytics, prevention and detection of diseases are in a new era of healthcare which essentially improving daily life of every patient. Personalize medicine is the in an era of new modern healthcare innovation. The role that big data analytics may have in interrogating the patient electronic health record headed for improved clinical decision support is discussed. In this paper we try to examine developments in pharmacogenetics that have enflamed our appreciation of the reasons why patients respond inversely to chemotherapy in cancer treatment. We also try to measure the development of online health infrastructures and the way healthcare data may be capitalized in order to detect public health warnings and control or comprise epidemics. Finally, this paper talks about how a new generation of body sensors in form of implanted in human body may improve comfort, rationalize management of chronic diseases and progress the superiority of surgical implants which could be effectively used in near future. So, let’s talk about what is precision medicine? How is it related to other dealings such as personalized medicine and omics technologies (especially in Pharmacogenomics and in Pharmacoproteomics)


\end{abstract}

\keywords{Keywords- HID348, Precision Medicine, Pharmacogenomics, Pharmacoproteomics. Big data Application, Data analytics, Big data infrastructure}


\maketitle

\section{Introduction}
The complexity, diversity, and rich context of data being generated in healthcare
are driving the development of big data for health [\cite{editor02}].The data captured at these portals can also help significantly reduce the cost of drug discovery as improved predictive analytics to determine which drugs work well and which are not as effective for certain conditions. Big data analytics may even allow for uploading the genomics of large populations that can be warehoused for researching new generations of drug remedies. Big data analytics is becoming increasingly popular in modern world with almost every domain. Big Data means lots of data used for analysis and get the insight from the data. Big data applications in general is applicable to any domain such as Retail, Healthcare, Finance and Supply Chain efficiently but in current world the application of Big Data has a major impact on healthcare sector where daily volume of data quadruple in every minute around the globe. Organizations are using Big Data to envisage the future with the goal of making them smarter and competitive in daily work. Applications from Big Data has become from retail industry where Big Data helps retailers gain insights into the customer needs and by monitoring customers’ habits future can be effectively utilized, HealthCare and Hospitality\cite{editor02}. Government agencies are progressively integrating Big Data analytics to control crime and sustain law by foresee the circumstances and by using social media, it is trying to achieve other benefits out of it. So, to get actionable data and perform analytics requires specialized tools which can handle this massive amount of data as well as help in analysis of the information. There are thousands of Big Data tools available in the market right now which contribute significantly to healthcare analytics. There are open source tools like Hadoop, which is named as big data umbrella and in big data ecosystem. Today’s healthcare data are beginning analyzed using aforesaid big data tools such as Pig, Cassendra, MongoDBand others. The quality of health care services in US and across the globe have been enhanced tremendously because of the advancement in health care services, advancements of technologies and Artificial intelligence process which improved the accuracy of healthcare as a service to next level. According to Google Trends analysis, the number of searches using the keyword ‘big data’ started to increase dramatically in 2011 and reached the peak 2017 \cite{editor02}. Although the term ‘big data’ resonances as if it is connected to the area of data science,  big data and data science both plays a significant role in healthcare research especially in precision and emergency medicine. Conventionally, scientists have adopted the traditional 4Vs criteria to describe big data: volume of data, velocity of data which is speed of incoming and outgoing data and variety which is range of data types \cite{editor03} However, from the perspective of medical science, this classification may not be real world sufficient as the 3Vs criteria are forceful and time reliant.
Big Data is a capacious collection of data that cannot be achieved by traditional database management systems (RDBMS). Big Data in an umbrella term used for the enormous amount of data produced from countless of sources such the mobile, web, sensor devices, enterprise applications and rigorous digital repositories.  In big data umbrella, data can be structured as well as unstructured or semi structured. The data varieties from terabytes to Exabyte of data \cite{editor03}.  The relational database management systems (RDBMS) have proven inefficient to handle such huge volumes of data in form of patient’s records such as X-ray, Scan and routine checkup results. . Another important factor which renders the conventional database systems inappropriate is that the majority of data being generated as unstructured, the RDBMS systems are only adept to handle structured relational data. Hereafter new tools and systems for data analysis and management have emerged. Volume, velocity, variety, veracity, variability, and value are the three must have Vs’ of big data and these are condensed in the integral challenges of biomedical and health informatics. Effective ways of confronting these challenges would cover the way for more intellectual healthcare systems focused on early detection, prevention and personalized treatments.  As Big data is characterized by the 4 V’s. We discussed about the 4V’s below which essentially contributed to the success of healthcare management\cite{editor02}.
1. Volume- As data are increasing day by day, it is always in light of voluminous collection of data. The complete volume of data generated these days by real time applications such as X-ray machines and MRI systems and other external data sources such as facebook comments, twitters or even patients data, it runs to petabytes and exabytes of data. Big Data technology empowers us to store this amount of data on dispersed systems[\cite{editor03}].
2. Velocity- is the proportion at which data is arrived. As an example, in whole gene sequencing process, one sequence generates huge volume of data and when the sequencing process is completed data arrives at a very higher speed having few time to store and analysis the data.
3. Veracity- When the volume increases so does the quality. Veracity refers to the quality of the data. There are doubt of good quality of accurate data being generated in recent times. Big Data applications empowers working with data which are large in volume, accurate and insightful\cite{editor03}.
4. Value- Everything in the world has a value so does the data. There is an intrinsic value that the data holds and discovers for analysis. Value is the heart of Big data analytics and the way data generated value in healthcare is just enormous. Modern technologies have made it possible to find the insight from data.  Since data is huge and storage capacity leads to an expensive turnaround. Apache Hadoop is the savor in these kind of applications by processing gigabytes of data in a very short span of times and Hadoop ecosystem consisting of Mapreduce a different language processing system or Hive and Drill an analytical SQL platform on Hadoop or Spark, in in memory data flow system or HBase/MongoDB in memory database systems or HDFS, capable of storing petabytes of data or streaming systems such as Apache Storm and Kafka, overall these all highly capable tools can be profoundly effective in healthcare biomedical data analytics.


\section{What is Precision Medicine}
Precision medicine in boarder terms is another name of preventive medicine According to the Precision Medicine Initiative and American Healthcare Association, precision medicine is an emerging approach for disease treatment and prevention that takes into interpretation of individual erraticism in genes, environment and routine and lifestyle for each person. This approach will allow medical professionals and researchers to predict more accurately about the treatment and prevention approaches for any particular disease about its effectiveness. It is in divergence to a one size approach in which disease treatment and deterrence strategies are developed for the average person with less deliberation for the genetics based differences between each individuals. Though the term precision medicine is relatively new in medical industry, the concept has been there and as a part of healthcare for many years. As an example, a patient who needs a blood transfusion is not given random blood from blood bank storage rather it will be from by a process of donor’s blood type is matched to the recipient to decrease the risk of future problems \cite{editor04}. Although illustrations can be found in numerous areas of medicine the role of precision medicine in day to day healthcare is relatively restricted. Medical researcher expect that this approach will increase in many areas of health and other healthcare domain in upcoming years. Precision medicine is been ought to transform how we as a whole improve health, treat and prevent disease. Today most of the medical treatments are intended for the average patient using the one and one approach. However, in many cases, this approach is not at all effective because treatments can be very successful for some patients but not for every people\cite{editor03}. As an example, if Patient A and Patient B both have stage 3 lung cancer , giving same chemotherapy to both the patient help one but not another. In precision medicine with the help of big data technology, medicines are targeted to specific genomic sequence rather a random selection.  In advanced counties like USA, a rigorous process is already in place to target particular gene after finding the root cause of the disease. It is big data application which enables to store the data and use analytical tool to get useful information out of it. Overall, Precision medicine is a field of medicine that takes into interpretation individual differences in people’s genes, environments , microbiomes, habitual effects, and family history to make diagnostic and beneficial strategies accurately personalized to individual patients. Precision medicine is a newer term referring to a similar ground compared to another word ‘personalized medicine’. The term ‘precision medicine’ arrived the scientific dictionary in the year 2008 when business strategist Clayton Christensen, of Harvard Business School in Boston, invented the appearance to describe how molecular diagnostics allows physicians to unambiguously diagnose the cause of a disease without having to rely on perception\cite{editor03}. The name precision medicine didn’t gain enough attention until 2011 when a committee convened by the US National Research Council placed out a plan for modernizing the classification of disease on the foundation of molecular information such as causal genetic variants instead of a symptom based cataloguing system. The committee called the report Toward Precision Medicine \cite{editor02}. There are many areas where precision medicine is vastly applicable and are very much beneficial such as, finding correct dose of prescription drugs, root cause analysis of a disease and so on. The field of pharmacogenomics aims to understand how genetic variations influence individual responses to medications. Genetic tests for supervisory treatment decisions are becoming increasingly available across miscellaneous areas of medical care. These kind of tests provide more effective drugs to patients earlier in their treatment and with fewer negative side effects and in less costly than previous tests. Precision medicine is also pertinent in Cancer detection, Genomics and cure process. Oncology is the target of some of the most auspicious precision medicine approaches available today. Cancer forms through the gradual accumulation of genetic DNA changes in genes that regulate cell growth. That is why, cancer is very much an illness of the genome. Depending on where in the body the cancer started and the types of genetic changes the cells grow, different types of cancer have very different genetic profiles which completely varies person to person and highly dependent on their family history. These genetic sequences can be used in a number of ways to help medical professional’s choosing the best treatments for each individual patient. Growing tissues replacement is another way to apply precision medicine in pharmacogenomics\cite{editor02}. 


\subsection{Personalized Medicine} The concept of personalized medicine dates back many hundreds of years although the term seemingly similar meaning with precision medicine. Mere from 19th century, scientists started to measure the chemistry of root cause of any illness and the research improvements are granular over time. With the growth of the pharmaceutical industry and medical technology industries in recent times came the rise of genetics, data mining and imaging. Halfway over the period, comments of specific alterations in retort to drugs contributed growth to a body of study attentive on classifying crucial enzymes that play an important role in disparity in drug absorption and reaction and this is helped as the basis for pharmacogenetics. In recent tims, sequencing of the human genome customary in motion the transformation of personalized medicine from an knowledge to a practice. Personalized diagnosed tools are now created with rapid developments in genomics along with advances high critical areas such as computational biology, medical imaging, and regenerative medicine and treatment \cite{editor03}. Personalized medicine first appeared in available mechanism in 1999 with the creation of some of the domain specific core concepts even dating back to 19th century \cite{editor05}.  So basically personalized medicine is referred to treatment depending on each individual’s personal structure and history. Initially, personalized medicine is the idea that assortment of a treatment should be custom-made giving to the individual patient’s specific physiognomies, including age, sex, gender, height, ethnicity, diet, and environmental factors against traditional clinical trials on group of people which has been happening since the invention of medicine happened \cite{editor03}. Scientists got interested on personalized medicine when medical professional started understanding the essence of gene in human development. Several human genome projects have been conducted since then and the importance of personalized medicine started in limelight. With deceitful out in order the 3.2 billion units of our DNA, scientists flashed a blaze of detection and a detonation of genomic knowledge in medical science history \cite{editor05}. Novel omics technologies including microarrays, whole genome single nucleotide polymorphism [SNP] chips, RNA interference high-throughput transmission, next generation sequencing are the few procedure which accompanied with this revolution. All the above launch a new epoch in personalized medicine which is called genomic revolution era which bids us limitless probable and countless promise containing the expansion of personalized medical products for each individual based on their sole genomic information \cite{editor04}. Advancement of genomics science along with the developing of new omics technologies, personalized medicine is today frequently well-defined as a combination of molecular profiling (omics methods) and customary methods such as family history, lifestyle and environment, which create analytic and beneficial strategies precisely personalized to individual patients \cite{editor05,editor03}.


\section{Big Data in Precision Medicine} Once again the term Big data is signifies in collection of large and complex data sets which are difficult or sometimes impossible to process using common database management tools or traditional data processing applications even with modern advancement of traditional data warehousing tools such as Amazon Redshift. In 2012, the Obama administration announced the Big Data Research and Development Initiative \cite{editor06}, which explored how big data could be utilized to address important problems faced by the overall healthcare system. Since then, Big Data becomes such a big term that people tend to claim any kind of data analysis to be ‘Big Data’ analysis. The overall concept of big data can be explained in various ways. One way is, Big data is a comprehensive term for any collection of data sets are so voluminous that processing the data in the begging stage itself is very hard.  With four ‘V’ characterization of big data m, complexity arises more to collect data and make meaningful information out of it. Omics data, mobile internet real-time data and electronic health record data are the top three areas for Big Data in medical research. Precision medicine will use all of these three Big Data. In fact, among the 215 million investment in the USA President’s 2016 Budget, 130 million (over 60  percent) will be used for building a large US cohort for precision research \cite{editor06}. In this regiment study, the scientists will use widespread omics data, electronic health record data gathered from several hospital and private practices along with mobile internet data [*]. Thus, omics and medical big data are one of the key pairs in the success of precision medicine in healthcare industry as a whole.


\section{Big data Challenges in Healthcare}- Whenever anything benefits us, that comes with its own challenges and problems. The primary idea of big data to be applied in healthcare is to roll massive healthcare dataset with individual information. As the need of more data driven enterprise grows Besides general challenges inherent to the analysis of big data such as missing data, vague data, and varied data, employing big data in health care systems imposes new challenges which includes the lack of reliability and a solid data governance of some biomedical data, issues of privacy and security and confidentiality, insufficient data from random clinical trials including successful and failed trails, and overall low quality data. Challenges in machine learning and statistical applications also put the analytics in challenging situation where model development and execution are critical to success\cite{editor05}. 
Healthcare providers who have hardly come to grasps with driving data into their electronic health records (EHR) are now being questioned to pull actionable insights out of them and apply those learnings to complex initiatives that straight impact their repayment rates. Organizations who can integrate this data driven technological innovation to their healthcare operations are in the most benefit\cite{editor07}. Data assets and data insights can be achieved by using healthier patients, increased visibility in operational excellence, lower care costs and higher staff and consumer satisfaction rates are among the many benefits of turning data assets into data insights. The journey to evocative healthcare analytics is difficult challenge and problems by solving those will benefit the industry to the highest extent. The way overall big data analytics work, collecting, storing, analyzing the data require clear presentation to the staff members to understand the overall workflow process\cite{editor03}. Analyzing genomic data is a computationally are some of the top challenges organizations typically aspect when striking up a big data analytics program and how can organizations overawed these issues to attain their data driven clinical and financial goals are the most important aspect of big data implementation. Understanding unstructured clinical nodes, storing unstructured patients health records are complex in nature and specialized training is required in implementing the analytics platform is essential. Some of the pitfall of big data application in precision medicine is discussed below.


\subsection{Data Collection} This is the most crucial stage in any data driven technologies, capturing the patient’s behavioral data through several sensing processes; with their numerous social interactions and communications.  The data many come from many sources or in different format but not everywhere data governance is properly applied while collecting the data. Capturing data which is clean, comprehensive, correct, and well formatted for use in diverse systems is an ongoing combat for organizations, many of which are not on the endearing side of the battle. As an example, electronic health record capturing in right movement help physicians to access the accurate picture of the patient’s history. Oftentimes, delay in collecting this data create problems which eventually leads to unhealthy environment and future risks. Revolving Healthcare Big Data into Actionable Clinical Intelligence Providers can start to recover their data capture procedures by ranking valuable data categories for their specific plans, conscripting the data governance and honesty knowledge of health information management professionals and evolving clinical documentation improvement programs that tutor clinicians about how to confirm that data is valuable for downstream analytics \cite{editor03}


\subsection{Data Cleaning} Healthcare providers are well familiar with the importance of cleanliness in the clinic and the operating room but they are not aware on many things which could lead to a clear picture of the meaningful data. Data which is dirty and raw might have a potential impact on big data analytics projects and can screw up the true insight completely.  
Data cleaning also known as data scrubbing always ensures that data is not inconsistent, proper and useful in perspective and predictive analytics. Though when everything started, data cleaning was a manual process, but now with the help of big data quality tools, cleaning data has been easier than ever before. Since data cleaning is complex and tedious process in particular healthcare system, oftentimes big data analytical tools stand by the first door where data streamline occurs. Which eventually cleans the data with a global standard before it entered to main stream pipeline.


\subsection{Data Storage} This is the most critical place where big data application play a key role.  As the volume of healthcare data grows exponentially many healthcare providers are not able to manage the costs and effects of on premise data centers. Although many organizations are most happy with on premise data storing which also leads to security issues and data governance issues. With the help of cloud storage almost 90 percent of healthcare providers have chosen cloud based data storage centers which provides better flexibility and availability of data. Amazon web services, Microsoft Azure cloud and Salesforce cloud have put the data storage industry to the utmost point where no longer organizations need to worry about the cost and capacity of storing humongous amount of data. The cloud offers sprightly disaster recovery, lower set up and upfront costs and easier development, although organizations must be extremely careful about choosing partners that understand the significance of HIPAA law and other healthcare related compliance and security issues \cite{editor02}. Many organizations finish up with an amalgam approach to their data storage agendas, which may be the furthermost supple and workable approach for providers with variable data access and storage necessities.  When creating hybrid substructure providers should be cautious to safeguard that dissimilar systems are able to interconnect and portion the data through extra segments of the organization when necessary \cite{editor07}.

\subsection{Data Security} Data security is the number one priority for healthcare organizations, particularly in the wake of a hundreds of data breaches, hackings, and intrusion incidents.  Data is so sensitive especially in healthcare systems that a proper security measure has to be taken to protect the data.  Healthcare privacy law such as HIPAA and others put the organizations in the front door where every healthcare providers must conform the law to protect the data. For precision based medicine era, this has become more important with each individual patient’s data being captured and analyzed. Since genomic science is completely depending on data architecture, one data breach can push the healthcare provider in a tremendous reputation and financial loss. Due to this, security is one of the most talked topic in personalized medicine \cite{editor05}.

\subsection{Data Governance} Healthcare data, particularly on the clinical side has a long ledge life.  In accumulation to existence required to keep patient data available for at least six years of time frame, providers might request to use de identified datasets for scientific projects, which makes continuing stewardship and curation an important concern \cite{editor05}.  Any data can be used for variety of other purposed as long as data masking is properly applied to the dataset. Understanding of the data when it is created by whom and for purpose can lead to positive results while in research. 

\subsection{Data Querying}
Vigorous metadata and robust stewardship procedures make organizations to comply with data querying very effectively. There are many tools in the market which can give access to query from databased to get the useful information. Azure datalakes, different programming based API, Hadoop Sqoop are the few tools which help in big data query language extraction.
Many organizations use Structured Query Language (SQL) to dive into large datasets and relational databases, but this can only be true if end user can trust the data they are working on which can provide useful information to them \cite{editor05}.
Data Reporting: Reporting is the end to end product of any data collection and process. Big data reporting can help transform virtually all aspects of the enterprise. From quickly producing actionable intelligence to driving productivity to gain real time visibility into customers and markets, big data analysis and big data reporting promise to deliver a wealth of benefits for competitive advantage [*]. Many companies including not for profit hospitals use reporting as their sole decision making procedure. Data reporting helps unveils the insight into charts and graphs and visualize the insight to the target audience. In healthcare organizations especially in precision medicines, reporting of the finding is must have to take informed decision. Data reporting has the potential to show about the result of an ongoing research study or data findings. \cite{editor05}.

\subsection{Data Visualization}
In patient care, a clean and attractive data visualization can make it much easier for a clinical staff to understand the fundamental very easily and take decision based on it. 
Color visualizations are a popular data visualization technique that typically yields an immediate response as an example, red, black color divergence is Organizations must also consider good data presentation practices such as charts, graphs, scatterplot. Common examples of data visualizations include heat maps, bar charts, pie charts, scatterplots, and histograms, all of which have their own specific uses to prove concepts and material.

\subsection{Data Update}Healthcare data is non-static and almost all the elements requires an update in daily interval.  Some datasets such as patient vital signs and symptoms, may require frequent update.  But patient’s demographic information may change once in a while. Since in genomic since changes are captured in every interval or procedure, there has to be constant update to the proper and existing dataset. The most critical phrase comes when incremental data is gathered and new data is added to existing dataset. In precision medicine, medical professional compares whole gene sequences in different timeframe of the disease and in different medicine stages. In these case, data has to be updated regularly in order to analyze the proper data \cite{editor0}.
Organizations should also confirm that they are not making needless identical records when endeavoring an update to a single component which may make it problematic for linical stff members to access needed information for patient decision making.


\subsection{Data Sharing}
With the essence of electronic medical record, data sharing become easier but complicated in healthcare analytics. With large volume and the structure of the data, healthcare providers and researchers are immensely beholding data which may contribute finding to their scientific invention. Data exchange is a perpetual worry for organizations at any costs. With the increase data volume and nature of the data, it is getting more and more difficult for the organization to move data from one to place to another without losing information and change in pattern on data lineage can lead to significant mislead information. \cite{editor07}. 



\section{Precision Medicine and Omics}
With the growth of big data, organizations more into NOSQL databases where security is a growing concern. Though we found there are severe security issues in most of the NOSQL databases which are used today in big data environment. Lack of security measures put extra sensitivity to the overall big data applications being NOSQL databases are heart of any big data project. Though not reached at pick, constant evaluation and research are in process to make NOSQL databases more secure in near future. The evolution of omics outlining technologies significantly benefited studies are conducted on diseases mechanism, molecular diagnosis and personalized treatment \cite{editor03}. The study of omics is strongly related to the study of biology as a whole and precision medicine. There is a strong connection between Omics and Precision medicine and big data as a whole has become the core of precision medicine. The advancement of precision/personalized medicine depends heavily on the ability to acquire biological aces at omics interval though the training of precision medicine does not use sole omics data and omics knowledge \cite{editor08}. This happens due to molecular characteristics found from omics data can categorize diseases and classify population of patients appropriate to assured common treatment more exactly \cite{editor03}.
Biology has become more data intensive and technological intensive subject .Following this trend, many of the emerging fields of large-scale data rich biology are designated by adding the suffix ‘-omics’ to previously used definitions. Particularly, the word omics refers to a field of study in biology ending in the suffix – omics and it is related addresses the objects of study of such a field\cite{editor03}\cite{editor05}.
Pharmacogenomics is the study of how a person’s response to drugs is affected by his genetic makeup \cite{editor02}. It combines pharmacology which is also called the science of drugs and genomics which is the study of gene and their functions to develop effective, proper medications that will be personalized to a person’s genetic makeup. Pharmacoproteomics, essentially a sub discipline of functional pharmacogenomics which is a study of how the protein content of a cell or tissue changes qualitatively and quantitatively in response to treatment or disease, what the protein-protein and protein ligand interactions are in related to drug response, and how a person’s protein variants in quality and quantity affect a person’s response to a drug \cite{editor04}. In modern days, the pharmaceutical industry has developed strong interest in Pharmacoproteomics with the anticipation that this technology will lead to the empathy and authentication of protein targets and eventually to the detection and growth of feasible drug candidates. Pharmacogenomics and Pharmacoproteomics will help the prescription of drug and related doses to a patients based on response to a drug which greatly indorsing the advance and practice of precision/personalized medicine \cite{editor04}

\section{Management and Processing of Omics data} There is no shortage of data in healthcare and it is growing 40 percent annually according to IDC. Data in healthcare is not always about volume in healthcare but there are several factors can contribute to it as well such as “Regulations”, “Complexity” and “integrity”. To process the volume and complexity of Omics data, there is a need of major investments in research laboratories in form of computational and storage capabilities. Laboratories need servers or cloud service storage access to store this massive amount of data. In traditional way, servers are costly in maintenance and ended up in profligate or sub optimal servers which are even more added cost load. In the past decade, cloud computing is closing the gap in hand ling omics data. Cloud computing is a high scalable multi-processing semantic environment which operate virtually with some of the great benefits to any organization such as costs, speed, global scale, productivity, performance and reliability. A good example comprise the ‘EasyGenomics’ cloud in Beijing Genomics Institute (BGI) and “Embassy” clouds as part of ELIXIR project in collaboration with multiple European countries (UK, Sweden, Switzerland, Czech Republic, Estonia, Norway, the Netherlands, and Denmark) \cite{editor10}. In several circumstances, Graphic Processing Units or GPUs ate also used in cloud environment as GPU’s are named to provide faster processing of data compared to Central Processing Units or CPU’s. There is also a high need of parallel computing in processing of Omics data along with data validation need in research platforms before Omics data are utilized. Along with software application in storage which is only one side of the medal there is a huge need of integration between biological system and the Omics data. Moreover, analysis of big data requires most recurrent data access to turn data into real knowledge. Even though we can take up the occurrence of an appropriate volume of bandwidth inside a cluster, there is a great need to have use distributed computing infrastructure in effective solution adaption. In big data technological umbrella there are two highest performance parallel file systems are the General Parallel File System (GPFS) \cite{editor10} a product of  IBM, and Lustre [] which is an open source platform. Predominantly most supercomputing systems such as Lustre is used in Titan, the second supercomputer of the TOP 100 list (December 2017) \cite{editor10}. The storage capability of Titan contains more than 20,000 disks which is equivalent to 40 Petabyte of storage and almost 1 Terabyte per second of storage bandwidth.  Among many software companies in big data business.  IBM file management solution is operational as a regulator plane for smooth data handling.  Since the access of data is so frequent, modern software’s can automatically switch less frequently accessed data to the less expensive storage available in the infrastructure keeping the most expensive storage for critical and sensitive data. Nowadays, moving the data from less expensive storage to expensive storage is completely depending on analytics supported decision making which includes pattern, storage characteristics and network pattern. Hadoop file systems plays a very important role in Omics data storage and overall data processing capabilities as explained above. Besides HDFS, Middleware is also essential in development of user specified custom solutions. A suitable example is R, considering a statistical programming tool, R is more robust now to handle high volume of data in biomedical data analytics and with help of middleware it can be used best in Omics data analysis as well.

\section{Analysis of Big data in Omics using Computational Facilities} HPC clusters along with grid computing used as customary platform for big data applications. Over the years it has been documented so many drawbacks of these platform in real environment where data manipulation and data integration were critical for success. Through cloud computing small and medium size laboratories can now leverage the power of data by accessing the data they want through cloud storage devices. Cluster computing which is a data parallel approach processes data independently with a high scalability. De Nova assembly algorithm is a renowned algorithm which is developed on cluster computing. This system can process daily genome sequence and find sequenced reads overlaps using in memory distributed approaches. Here the data in held in memory clusters unless it is sequenced differently \cite{editor11}. Another high computing big data application is Intel’s Xeon Phi which can deliver massive parallelism and vectorization to support high performance computing applications. Xeon Phi is an excellent support systems application in data discovery workloads and high dimensional matrices can be used to a level which can significantly benefit the thread level parallelism. Along with some established and newly added big data application which can great change the biomedical industry, we have documented the used of data semantics in many research laboratories in their data discovery process. Semantic data uses RDF or Resource description framework, Web Ontology Language or OWL, SPARQL or Protocol and query language for semantic web data sources and extensively use of XML (Extensible Markup Language). Oftentimes these have a dense use in bioinformatics and biostatistics to integrate all data formats and standardize existing ontologies \cite{editor11}. 

\section{Conclusion} Personal medicine, Omics technologies and pharmacogenomics are the evolutionary invention in medical industry, holding the hands of these medical concepts scientist can not only find the cancer cell in human body parts as well as start of cancer as a disease in a particular cell. These all is possible due to the essence of big data which not only helped organizations to tackle the voluminous data effectively but to use them in a way to get meaningful insight out of it. Massive parallel computing and clustering are now opened up new window in medical research where processing of huge amount data is better than ever before as well as build automated model on top of it. Whole gene sequencing is an example of how a big data can help strong millions of genetic information in a single storage system and take useful information out of it. With the help of big data in precision based medicines scientists are now able to predict the origin of the disease, track and cure it more effectively.

\begin{acks}

  The author would like to thank Dr. Gregor von Laszewski and I523.

\end{acks}


\bibliographystyle{ACM-Reference-Format}
\bibliography{report} 

\end{document}
